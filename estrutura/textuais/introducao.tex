% INTRODUÇÃO-------------------------------------------------------------------

\chapter{INTRODUÇÃO}
\label{chap:introducao}

Uma pesquisa realizada pela \textit{Newzoo}, uma empresa de consultoria e pesquisas, mostrou que no ano de 2018 o Brasil tinha 75,5 milhões de jogadores de jogos eletrônicos, com isso o Brasil é o 13º mercado do mundo gerando em torno de US\$1,5 bilhões por ano \cite{NEWZOO18}.Tendo isso em vista pode se afirmar que o mercado de \textit{games} tem emergido e se tornado um dos mais relevantes no Brasil. 


Dentre tantos tipos de jogos, temos os jogos educacionais, que são jogos utilizados como ferramenta para ensinar um conceito sobre algum assunto desejado, seja para crianças, jovens, adultos e até mesmo idosos \cite{Rabin2009}, temos exemplos de jogos educativos muito antes dos jogos se popularizarem, como é o exemplo do jogo \textit{The Oregon Trail} que tinha como objetivo ensinar as crianças sobre a história dos trilhos de \textit{Oregon}, foi lançado no ano de 1971 mas só se popularizou nas décadas de 80 e 90 graças a uma versão feita para os computadores \textit{Apple} que eram presentes nas escolas. Jogos educativos são uma forma de se ensinar algo de maneira a instigar o interesse do aluno, que muitas vezes não se sente desafiado simplesmente preenchendo lacunas em um livro, então com um jogo tudo se torna mais concreto e o processo de aprendizado se torna mais prazeroso e interessante \cite{Joceline2006}. Os jogos educacionais visam suprir a falta de um ambiente de aplicação do que é aprendido\cite{Valente98}, no caso do jogo que será desenvolvido para ensinar inglês, nem todos tem a possibilidade de fazer um intercâmbio ou viajar para o exterior para aprender e aplicar diariamente a língua inglesa, por isso o jogo visa uma aplicação mais prática da língua.

Foi escolhido para o trabalho ensinar a língua inglesa pois as empresas e universidades tem cada vez mais ingressado na globalização e o inglês é a língua comercial e das tecnologias, portanto para conseguir ingressar em uma universidade ou no mercado de trabalho é imprescindível o conhecimento da língua inglesa. Quanto antes as pessoas criarem uma base fundamentada do inglês, mais fácil será o aprendizado e a familiarização com o novo idioma por isso é importante que as crianças tenham esse contato com a língua \cite{Medina2013}, e uma forma de criar esse interesse desde cedo é por meio de um jogo.

Segundo uma pesquisa realizada por \cite{Cruz2012}, em uma amostra de 322 pessoas com média de 13 anos, 99\% delas tinham computador em casa, isso nos mostra que dificilmente nos lares brasileiros de hoje em dia não se tem pelo menos um computador, e quando for o caso de não terem, as escolas públicas hoje em dia também são equipadas com computadores, então o jogo seria uma ferramenta que alcançaria muitas pessoas interessadas, pois seria distribuído de forma gratuita.  




\section{OBJETIVO}
\label{sec:objetivo}

Objetivo aqui

\section{JUSTIFICATIVA}
\label{sec:justificativa}

Justificativa vai aqui!

\section{ORGANIZAÇÃO DO TRABALHO}
\label{sec:organizacaoTrabalho}

Normalmente ao final da introdução é apresentada, em um ou dois parágrafos curtos, a organização do restante do trabalho acadêmico.
Deve-se dizer o quê será apresentado em cada um dos demais capítulos.
