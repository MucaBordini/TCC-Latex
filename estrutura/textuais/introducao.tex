% INTRODUÇÃO-------------------------------------------------------------------

\chapter{INTRODUÇÃO}
\label{chap:introducao}

Uma pesquisa realizada pela \textit{Newzoo}, uma empresa de consultoria e pesquisas, mostrou que no ano de 2018 o Brasil tinha 75,5 milhões de jogadores de jogos eletrônicos, com isso o Brasil é o 13º mercado do mundo gerando em torno de US\$1,5 bilhões por ano \cite{NEWZOO18}.
Tendo isso em vista pode se afirmar que o mercado de \textit{games} tem emergido e se tornado um dos mais relevantes no Brasil. 
Mas ainda existe a ideia de que jogos são coisa de criança, pois uma pesquisa realizada pela Pesquisa Game Brasil 2019 mostrou que isso é bem diferente. A pesquisa foi dividida entre: Jogadores casuais, os quais jogam de forma esporádica como simples passatempo, sem muito comprometimento com o jogo. E os jogadores \textit{hardcore} que são os jogadores que já levam o ato de jogar mais a sério, se dedicando horas para criar estratégias, estudando o jogo e muitos até se profissionalizando, participando de competições de \textit{e-sports}, que são campeonatos de jogos eletrônicos.
\section{OBJETIVO}
\label{sec:objetivo}

Objetivo aqui

\section{JUSTIFICATIVA}
\label{sec:justificativa}

Justificativa vai aqui!

\section{ORGANIZAÇÃO DO TRABALHO}
\label{sec:organizacaoTrabalho}

Normalmente ao final da introdução é apresentada, em um ou dois parágrafos curtos, a organização do restante do trabalho acadêmico.
Deve-se dizer o quê será apresentado em cada um dos demais capítulos.
