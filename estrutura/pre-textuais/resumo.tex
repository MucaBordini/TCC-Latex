% RESUMO--------------------------------------------------------------------------------

\begin{resumo}[RESUMO]
\begin{SingleSpacing}

% Não altere esta seção do texto--------------------------------------------------------
\imprimirautorcitacao. \imprimirtitulo. \imprimirdata. \pageref {LastPage} f. \imprimirprojeto\ – \imprimirprograma, \imprimirinstituicao. \imprimirlocal, \imprimirdata.\\
%---------------------------------------------------------------------------------------

Jogos eletrônicos tem se popularizado muito nos dias atuais, cada vez mais a industria de \textit{games} tem se tornado algo relevante, fazendo com que os jogos deixem de ser algo infantil para se tornarem um mercado enorme, competindo até mesmo com o cinema.

Tendo isso em vista, este trabalho terá como foco desenvolver um jogo de plataforma 2D do gênero aventura para o ensino de vocabulários da língua inglesa. Seu público alvo serão pessoas que desejam estudar o básico da língua inglesa e também que simplesmente busca diversão. O jogo terá uma história linear, que utilizará algumas mecânicas de usar palavras para progredir no jogo, mas tudo isso de uma forma ficcional, com inimigos que irão atrapalhar seu objetivo e elementos da história que você precisará superar resolvendo quebra-cabeças. Foi pensado desta maneira pois o foco do jogo deve ser sempre a diversão, ensinar algo se torna uma consequência, o jogo se trata da experiência de jogar, quanto mais prazerosa for, mais conteúdo o jogador irá assimilar. Se possível também, ao final do desenvolvimento, o jogo será publicado em certas plataformas. A ideia inicial é desenvolve-lo para \textit{Windows} e realizar a portabilidade para celulares com o sistema \textit{Android}, tudo isso utilizando o motor gráfico \textit{Unity} pois é uma ferramenta que possui uma versão gratuita com todas funcionalidades necessárias para se implementar um jogo 2D.


\textbf{Palavras-chave}: Ensino. Jogo. 2D. Aventura. Inglês. Entretenimento.

\end{SingleSpacing}
\end{resumo}

% OBSERVAÇÕES---------------------------------------------------------------------------
% Altere o texto inserindo o Resumo do seu trabalho.
% Escolha de 3 a 5 palavras ou termos que descrevam bem o seu trabalho 
