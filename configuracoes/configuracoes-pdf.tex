% CONFIGURAÇÕES DE APARÊNCIA DO PDF FINAL--------------------------------------
\makeatletter
\hypersetup{%
    portuguese,
    colorlinks=true,   % true: "links" coloridos; false: "links" em caixas de texto
    linkcolor=blue,    % Define cor dos "links" internos
    citecolor=blue,    % Define cor dos "links" para as referências bibliográficas
    filecolor=blue,    % Define cor dos "links" para arquivos
    urlcolor=blue,     % Define a cor dos "hiperlinks"
    breaklinks=true,
    pdftitle={\@title},
    pdfauthor={\@author},
    pdfkeywords={abnt, latex, abntex, abntex2}
}
\makeatother

% ALTERA O ASPECTO DA COR AZUL--------------------------------------------------
\definecolor{blue}{RGB}{41,5,195}

% REDEFINIÇÃO DE LABELS---------------------------------------------------------
\renewcommand{\algorithmautorefname}{Algoritmo}
\def\equationautorefname~#1\null{Equa\c c\~ao~(#1)\null}

% CRIA ÍNDICE REMISSIVO---------------------------------------------------------
\makeindex

% HIFENIZAÇÃO DE PALAVRAS QUE NÃO ESTÃO NO DICIONÁRIO---------------------------
\hyphenation{%
    qua-dros-cha-ve
    Kat-sa-gge-los
}
